\documentclass[12pt]{article}

\usepackage{sbc-template}

\usepackage{graphicx,url}

%\usepackage[brazil]{babel}   
\usepackage[latin1]{inputenc}  

\usepackage{amssymb}
\usepackage{amsmath}
\usepackage{mathtools}
\usepackage{hyperref}
\usepackage{algorithm}
\usepackage{algorithmic}

\algsetup{linenosize=\small,linenodelimiter=.}
\hypersetup{colorlinks,citecolor=black,filecolor=black,linkcolor=black,urlcolor=black}

\usepackage{amsthm}

\newtheorem{definicao}{Defini��o}
\newtheorem{exemplo}{Exemplo}
     
\sloppy

\title{Normaliza��o de ontologias em l�gica de descri��es para o racioc�nio com o provador de teoremas leanCoP}

\author{Adriano S. T. Melo\inst{1}, Fred Freitas\inst{1}}


\address{Centro de Inform�tica - Universidade Federal de Pernambuco\\
  Caixa Postal 50740-560 -- Cidade Universit�ria -- Recife -- PE -- Brasil
  \email{\{astm,fred\}@cin.ufpe.br}
}

\begin{document} 

\maketitle

\begin{abstract}
The Semantic Web promises to revolutionize the way people interact on the Web. Concepts like intelligent agents, semantic-based services, business-to-business communication between companies through rule-based agents should be plentiful in this new scenario. Several technologies are being studied and constructed to form a new infrastructure for the Web. These are the elements that form the layer of logic and proof, context that this project is inserted. This project proposes to implement part of the algorithm of the connection method for description logic, used to do activities of reasoning in a knowledge base. The module to be implemented is the conversion of the axioms of the knowledge base for the positive normal form.
\end{abstract}
     
\begin{resumo} 
 A Web Sem�ntica promete revolucionar o jeito de como as pessoas interagem na Web. Os conceitos de agentes inteligentes, servi�os baseados em sem�ntica de documentos e empresas se comunicando com empresas atrav�s de agentes baseados em regras dever�o ser abundantes nesse novo cen�rio. V�rias tecnologias est�o sendo estudadas e constru�das para formar uma nova infra-estrutura para a Web. Entre elas est�o os elementos que formar�o a camada de l�gica e prova, contexto que est� inserido este trabalho. Este trabalho prop�e implementar parte do algoritmo do m�todo das conex�es para l�gica de descri��o, usado para fazer atividades de racioc�nio em uma base de conhecimento.  O m�dulo a ser implementado � a convers�o dos axiomas da base de conhecimento para a forma normal positiva.
\end{resumo}

\section{Introdu��o}
% !TEX encoding = ISO-8859-1
%\chapter{Introdu��o}
%\label{ch:introducao}

%\section{Contextualiza��o}
A World Wide Web � uma das tecnologias mais revolucion�rias que o homem j� inventou. Ela mudou em escala global a forma com que pessoas e empresas trocam informa��es, contribuindo para que o conhecimento se tornasse mais universal e que limites f�sicos e ling��sticos fossem cada vez mais minimizados.

%Tim Berners-Lee, o inventor da Web, tinha preten��es bem definidas ao elaborar essa infra-estrutura, seu principal objetivo era melhorar a forma com que o conhecimento era mantido dentro de uma empresa, e nessa concep��o, o que importaria era a visualiza��o dos dados por humanos, e n�o pelas m�quinas. Hoje, a preocupa��o � outra, como estruturar melhor as informa��es que s�o abundantes na Web para que sejam compreendias pelas m�quinas?

A web como conhecemos hoje nasceu de uma proposta feita por Tim Berners-Lee � empresa CERN em 1989 \cite{BernersLee:1989}. O problema enfrentado pela empresa na �poca era a perca de informa��es internas por falta de documenta��o ou pela sa�da de algum funcion�rio. A solu��o proposta por Berners-Lee foi fazer uma rede de documentos interligados por hyperlinks em que cada setor da empresa poderia adicionar novos documentos. 

A estrutura b�sica que Berners-Lee montou a 22 anos evoluiu a passos largos em rela��o � escalabilidade e padroniza��o de protocolos e linguagens, tendo hoje cerca de 2 bilh�es de usu�rios, mais de 30\% da popula��o do planeta. 

Apesar do avan�o das infra-estruturas e servi�os para a Web, ainda h� muito o que evoluir. Uma das propostas de mudan�as � prover uma maior expressividade da linguagem que descreve os documentos na Web \cite{Heflin:2004}. Hoje, esses documentos n�o possuem um significado que possa ser extra�do de forma concisa, apresentam ambig�idade, misturam os dados com elementos visuais e muitas vezes n�o podem ser indexados por engenhos de busca.

\section{Web Sem�ntica}

\begin{quote}
\textit{"I have a dream for the Web [in which computers] become capable of analyzing all the data on the Web, the content, links, and transactions between people and computers. A \textbf{Semantic Web} which should make this possible has yet to emerge, but when it does, the day-to-day mechanisms of trade, bureaucracy and our daily lives will be handled by machines talking to machines. The \textbf{intelligent agents} people have touted for ages will finally materialize." Tim Berners-Lee}
\end{quote}

\begin{quote}
Tradu��o literal: \textit{"Eu tenho um sonho para a Web [em que os computadores] tornam-se capazes de analisar todos os dados na Web, o conte�do, links, e as transa��es entre pessoas e computadores. A \textbf{Web Sem�ntica} que deve tornar isso poss�vel ainda est� para surgir, mas quando isso acontecer, os mecanismos dia-a-dia da burocracia do com�rcio e nossas vidas di�rias ser�o tratados por m�quinas falando com m�quinas. Os \textbf{agentes inteligentes} que as pessoas t�m falado por anos v�o finalmente se concretizar." Tim Berners-Lee}
\end{quote}

A Web Sem�ntica citada no texto de Berners-Lee acima � uma iniciativa de pesquisadores da �rea de intelig�ncia artificial e ling��stica computacional que estudam como adequar a Web de hoje a uma infra-estrutura que a tornar� mais acess�vel �s m�quinas. Essa nova roupagem que os pesquisadores querem dar � Web permitir� que servi�os mais sofisticados possam ser constru�dos, como os que ser�o descritos a seguir.

\subsection{Aplica��es}

\subsubsection{Gerenciamento de Conhecimento}
%\textbf{Gerenciamento de Conhecimento}

Gerenciamento de conhecimento est� relacionado � aquisi��o, acesso e manuten��o de conhecimento dentro de uma empresa ou organiza��o. Essa atividade se tornou e est� se estabelecendo como uma necessidade b�sica em grandes empresas visto que o conhecimento que � gerado internamente agrega valor, pode se tornar um diferencial competitivo e tamb�m pode aumentar a produtividade de seus colaboradores. Com o uso de tecnologias criadas para a Web Sem�ntica, solu��es para G.C. podem melhorar em v�rios aspectos, entre eles:

\begin{itemize}
\item Organiza��o do conhecimento existente a partir de seu significado;
\item Gera��o de novas informa��es de forma autom�tica;
\item Checagem de inconsist�ncias sem�nticas em documentos;
\item Substitui��o de consultas baseadas em palavras-chave por perguntas em linguagem natural; 
\end{itemize}

\subsubsection{Com�rcio eletr�nico \textit{Business to Consumer} (B2C)}
%\textbf{Com�rcio eletr�nico Business to Consumer (B2C)}

O com�rcio eletr�nico entre vendedores e consumidores � um dos modelos de neg�cio na Internet que melhor se estabeleceu, sites como amazon \footnote{site: amazon.com}, americanas \footnote{site: americanas.com} e mercado livre \footnote{site: mercadolivre.com.br} possuem p�blico fiel e que os visitam por v�rios objetivos. � muito comum para a gera��o que cresceu imersa na Web entrar em sites de compra como esses a procura do melhor pre�o antes de decidir fazer uma compra. Muitas vezes o produto n�o � adquirido em uma loja virtual, mas a pesquisa inicial de pre�os � que muitas vezes determina a escolha do produto. Observando esse comportamento, sites como o buscap� \footnote{site: buscape.com.br} fazem o trabalho de indicar qual � a loja que est� com o melhor pre�o. 

A Web Sem�ntica pode ajudar nesse cen�rio provendo interfaces de consulta mais completas aos sites que fazem compara��o de pre�os, por�m, com muito mais detalhes t�cnicos sobre o produto. Supondo que cada produto tem, por exemplo, uma ontologia que o descreve em detalhes (provida pelo fabricante ou por sites de review de produtos), o consumidor poder� fazer compara��es muito mais detalhadas, ajudando-o a encontrar o produto que vai suprir a sua necessidade.

\subsubsection{Com�rcio eletr�nico \textit{Business to Business} (B2B) e agentes pessoais}
%\textbf{Com�rcio eletr�nico Business to Business (B2B) e agentes pessoais}

A maioria das pessoas que compram servi�os ou produtos na Web s� conhecem o com�rcio eletr�nico do tipo B2C, mas existem tecnologias para com�rcio do tipo B2B, Business to Business, agentes computacionais de empresas que se comunicam para fechar acordos e otimizar o ciclo de neg�cios que muitas vezes j� podem ser previstos e modelados.

Com a populariza��o da Web Sem�ntica e a introdu��o de agentes pessoais e agentes que representam neg�cios, eles poder�o se comunicar de forma mais natural e aplica��es para otimizar tarefas do dia-a-dia que poder�o ser produzidos. Por exemplo, um m�dico que possua um agente pessoal que negocie a sua agenda com os agentes pessoais de seus clientes pode ser utilizado para remarcar seus atendimentos em caso de uma viagem ou imprevisto, agindo como uma secret�ria virtual.

\subsection {Tecnologias}

Aplica��es como as citadas acima j� existem, mas o trabalho de engenharia para conseguir bons resultados � alto devido �s tecnologias que s�o adotadas hoje. Vamos usar o case do \textit{site} de compara��o de pre�os BuscaP� na pr�xima se��o da monografia. 

\subsubsection{Metadados Expl�citos}

A primeira tarefa de engenharia de um agente coletor de pre�o, como os do buscaP�, � faz�-lo visitar v�rios sites de compras todo dia � procura de modifica��es nas listas de produtos para saber quais est�o dispon�veis naquela loja. � feito ent�o o \textit{parsing} do HTML de cada site de compras � procura das informa��es de pre�o, descri��o, avalia��o e detalhes de cada produto. A limita��o dessa abordagem � que sempre que um dos sites de compra mudar o \textit{layout} (estrutura do HTML), um novo script de \textit{parsing} dever� ser escrito. A grande demanda t�cnica de uma aplica��o como essa � a escrita de agentes muito especializados para atingir bons resultados.

Na figura 1.1 est� parte do c�digo em HTML de uma p�gina de produtos da americanas.com. As informa��es do produto est�o cercados apenas de c�digo para a renderiza��o dessa p�gina pelo navegador. Ou seja, a �nica preocupa��o dos engenheiros da americanas.com foi a leitura por humanos da informa��o do produto. Uma aplica��o que deseje usar as informa��es dos produtos da loja vai ter que fazer um agente especializado no \textit{parsing} desse c�digo.

\begin{figure}
\includegraphics[scale=0.55]{../imagens/americanas.png}
\caption{C�digo HTML de uma p�gina da americanas.com}
\label{fig:americanas}
\end{figure}

Motores de busca que tamb�m se baseiam em \textit{parsing} de p�ginas para extrair informa��es da Web dificilmente saber�o, por exemplo, qual � o pre�o de um produto nesse site da americanas.com, j� que h� v�rias informa��es de pre�o na p�gina e o \textit{parsing} que � feito n�o � otimizado para sites espec�ficos.

A conseq��ncia para o usu�rio final � que ele ter� que usar um site espec�fico como o buscap� ou ter� que fazer buscas a um engenho de busca por palavras-chave para achar os sites de compra que possuam um produto e em uma segunda etapa, fazer a an�lise de pre�os manualmente.

A abordagem da Web Sem�ntica para resolver problemas como esse n�o � fazer agentes especializados (como os do buscap�), e sim, anotar metadados sem�nticos dos documentos dispon�veis na Web. O exemplo dado anteriormente seria escrito na figura 1.2.

\begin{figure}
\includegraphics[scale=0.55]{../imagens/americanas-xml.png}
\caption{Exemplo de c�digo com sem�ntica para um produto}
\label{fig:americanasxml}
\end{figure}

\subsubsection{Ontologias}

O termo ontologia vem da filosofia, nesse contexto, � um ramo da filosofia que se dedica a estudar a natureza da exist�ncia, concentra-se em identificar e descrever o qu existe no universo. Em computa��o, uma ontologia � um artefado para descrever um dom�nio. Consiste em uma lista finita de termos e rela��es entre eles. Os termos denotam conceitos importantes de um dom�nio \cite{Antoniou:2008}.

Grande parte dos trabalhos referentes � Web Semantica est�o ligados a ontologias, inclusive este. As linguagem de descri��o de ontologias mais import�ntes para a Web s�o:

\begin{itemize}
\item XML: usado para dirigir a sintaxe de documentos estruturados. N�o imp�e restri��es sem�nticas no conte�do do documento;
\item XML Schema: linguagem para impor restri��es na estrutura dos documentos XML;
\item RDF: modelo de dados para recursos (objetos) e rela��es entre eles. As restri��es sem�nticas s�o fixas e podem ser representados a partir da sintaxe do XML;
\item RDF Schema: descreve as propriedades e classes dos objetos RDF;
\item OWL: linguagem rica para modelagem de classes, propriedades, rela��es entre classes (e.g. disjun��o), restri��es de cardinalidade, caracter�sticas de propriedades (e.g. simetria). Mais detalhes sobre a OWL ser�o dados no cap�tulo 2 dessa monografia.
\end{itemize}

\subsubsection{L�gica}

L�gica � a disciplina que estuda os princ�pios do racioc�nio. Ela prov� linguagens formais para expressar conhecimento, a sem�ntica formal para a interpreta��o de senten�as sem precisar realizar opera��es sobre a base de conhecimento e a transforma��o de conhecimento impl�cito em conhecimento expl�cito, atrav�s de dedu��es a partir da base de conhecimento \cite{Antoniou:2008}.

L�gica � mais geral que ontologias, ela pode ser usada por agentes inteligentes para tomada de decis�es e escolha de a��es. Por exemplo, um agente de B2C pode dar um desconto a um cliente baseado na seguinte regra:

\[
\forall x \forall y, cliente(x) \land produto(y) \land clienteFiel(x) \rightarrow desconto(x, y, 5\%)
\]

Onde \textit{cliente(x)} indica que x � um cliente/consumidor, \textit{produto(y)} indica que y � um produto de uma loja, \textit{clienteFiel(x)} indica que x � um cliente fiel da loja e \textit{desconto(x, y, 5\%)} indica que o cliente x ter� um desconto de 5\% no produto y.

\subsubsection{Agentes}

Um agente � tudo o que pode ser considerado capaz de perceber seu ambiente por meio de sensores e de agir sobre esse ambiente por meio de atuadores \cite{AIMA}. Agentes l�gicos s�o aqueles que executam a��es atrav�s de uma base de conhecimento e possuem um requisito fundamental, quando ele formula uma pergunta para a base de conhecimento, a resposta deve seguir o que j� foi informado anteriormente.

Agentes para a Web Sem�ntica utilizam as tr�s tecnologias que j� foram descritas:

\begin{itemize}
\item Metadados ser�o usados para identificar e extrair informa��es da Web;
\item Ontologias ser�o usadas para dar assist�ncia �s consultas realizadas � Web, interpretar informa��es recuperadas e para comunica��o com outros agentes;
\item L�gica ser� usada para processar informa��es recuperadas, chegar a conclus�es e tomar decis�es;
\end{itemize}

\section{Organiza��o da Monografia}

Esta monografia est� dividida em cinco cap�tulos. No Cap�tulo 1, � apresentada uma vis�o geral sobre a Web Sem�ntica, exemplificando com aplica��es e citando as tecnologias que est�o sendo usadas. No Cap�tulo 2, s�o apresentados conceitos referentes a L�gica de Descri��o e a sua liga��o com a linguagem de descri��o de ontologias OWL. No Cap�tulo 3 s�o descritos os algoritmos para normaliza��o de ontologias para a Forma Normal Positiva. No Cap�tulo 4 o LeanCop � apresentado e � mostrada a valida��o do trabalho realizado. O Cap�tulo 5 apresenta as considera��es finais sobre o trabalho, bem como propostas de trabalhos futuros.

\section{L�gica de Descri��o $\mathnormal{ALC}$}
% !TEX encoding = ISO-8859-1
\chapter{L�gica de Descri��o ALC}
\label{ch:logicadedescricaoalc}

L�gica de Descri��o � uma fam�lia de linguagens de representa��o de conhecimento que pode ser usada para representar o conhecimento de dom�nio de uma aplica��o de forma estruturada e formal \cite{dlhandbook:2003}.

A motiva��o para estudar l�gica de descri��o neste trabalho vem da Web Sem�ntica. Para que as m�quinas possam fazer infer�ncias sobre os documentos da Web, � preciso que a linguagem de descri��o dos documentos v� al�m da sem�ntica b�sica definida pelo RDF Schema e consiga definir e descrever classes e propriedades sobre os objetos encontrados na Web.

%A Web Ontology Language (OWL) � a linguagem de ontologias baseada em l�gica de descri��o especificada pela W3C que dar� esse passo evolutivo � Web em dire��o � Web Sem�ntica.

\section{Sintaxe da L�gica de Descri��o}
Nessa se��o ser� mostrada a sintaxe b�sica da l�gica de descri��o. A tabela 2.1 mostra o alfabeto de s�mbolos usado pela linguagem.

\begin{table}
\begin{center}
\begin{tabular}{ | l | l | } \hline
\multicolumn{2}{|c|}{alfabeto} \\ \hline
a, b 	& indiv�duos \\ \hline
A, B 	& conceitos at�micos \\ \hline
C, D 	& descri��o de conteitos \\ \hline
R, S 	& papeis (propriedades) \\ \hline
f, g 	& s�mbolos de fun��es\\ \hline
\multicolumn{2}{|c|}{conectivos} \\ \hline
$\sqcap$ & interse��o \\ \hline
$\sqcup$ & uni�o \\ \hline
$\neg$ & nega��o \\ \hline
\multicolumn{2}{|c|}{rela��es} \\ \hline
$\sqsubseteq$ & inclus�o \\ \hline
$\equiv$ & equival�ncia \\ \hline
\end{tabular}
\caption{Nota��o da l�gica de descri��o}
\end{center}
\end{table}

Os elementos mais b�sicos s�o os conceitos at�micos e propriedades at�micas. Descri��es de conceitos podem ser contru�das indutivamente a partir dos contrutores com conceitos e propriedades.

\begin{center}
\begin{tabular}{ l l l }
$ C, D \rightarrow$ 	& $A$ | 			& (conceito at�mico) \\
 				& $\top$ |		& (conceito universal) \\
 				& $\bot$ |		& (conceito vazio) \\
 				& $\neg A$ |	& (nega��o de conceito at�mico) \\
 				& $C \sqcap D$ |	& (interse��o de conceitos) \\
 				& $\forall R.C$ |	& (restri��o de valor) \\
 				& $\exists R.\top$	& (restri��o existencial) \\
\end{tabular}
\end{center}

Uma intepreta��o $\iota$ consiste em um conjunto n�o vazio $\Delta^\iota$ (dom�nio da interpreta��o) e uma fun��o de interpreta��o, que para conceito at�mico $A$ � o conjunto $A^\iota \subseteq \Delta^\iota$ e para cada propriedade at�mica $R$ � a rela��o bin�ria $R^\iota \subseteq \Delta^\iota \times \Delta^\iota$. As fun��es de interpreta��o se extendem a descri��o de conceitos a partir das defini��es indutivas \cite{dlhandbook:2003} como as que est�o abaixo:

\begin{center}
\begin{tabular}{ r c l }
$\top^\iota$ 		&	$=$	&	$\Delta^\iota$ \\
$\bot^\iota$		&	$=$	&	$\emptyset$ \\
$\neg A^\iota$ 		&	$=$	&	$\Delta^\iota \backslash A^\iota$ \\
$(C \sqcap D)^\iota$ 	&	$=$	&	$C^\iota \cap D^\iota$ \\
$(\forall R.C)^\iota$ 	&	$=$	&	$\{ a \in \Delta^\iota | \forall b.  (a, b) \in R^\iota \rightarrow b \in C^\iota \}$ \\
$(\exists R.\top)^\iota$ 	&	$=$	&	$\{ a \in \Delta^\iota | \exists b.  (a, b) \in R^\iota \}$ \\
\end{tabular}
\end{center}

Esse trabalho � restrito � famila ALC, que compreende os conceitos e propriedades at�micas, nega��o de conceitos, interse��o, uni�o, restri��es de valor e existencial, \textit{top} (verdade) e \textit{bottom} (absurdo). A tabela 2.2 mostra al�m de ALC, outras fam�lias de DL existentes \cite{dlhandbook:2003}.

Uma ontologia ou base de conhecimento em ALC � composta pela tripla $(N_C, N_R, N_O)$, onde $N_C$ � o conjunto de conceitos, $N_R$ � o conjunto de predicados, e $N_O$ � o conjunto de indiv�duos, que s�o as inst�ncias de $N_C$ e $N_R$. A base de conhecimento ou ontologia tamb�m pode ser descrita como o par $(\tau, \alpha)$, onde $\tau$ � a terminologia do dom�nio (TBox), equivalente a $N_C \cup N_C$ e $\alpha$ � a instancia��o da base, que corresponde a $N_O$, tamb�m conhecida como \textit{assertional box} (ABox).

Os axiomas s�o compostos por elementos de $N_O$ e um conjunto finito de GCIs (\textit{general concept inclusions}). Podem assumir a forma $C \sqsubseteq D$ ou $C \equiv D$ (uma equivalencia ($\equiv$) � o mesmo que $(C \sqsubseteq D) \land (D \sqsubseteq C)$ ), onde $C, D$ s�o conceitos e $\sqsubseteq$ � uma inclus�o.

\begin{table}
\begin{center}
\begin{tabular}{ l l l r }
\hline
Nome & Sintaxe & Sem�ntica & Expressividade \\ \hline
Verdade & $\top$ & $\Delta^\iota$ & $AL$ \\ \hline
Absurdo & $\bot$ & $\emptyset$ & $AL$ \\ \hline
Interse��o & $C \sqcap D$ & $C^\iota \cap D^\iota$ & $AL$ \\ \hline
Uni�o & $C \sqcup D$ & $C^\iota \cup D^\iota$ & $U$ \\ \hline
Nega��o & $\neg C$ & $\Delta^\iota \backslash A^\iota$ & $C$ \\ \hline
Restri��o de valor & $\forall R.C$ & $\{ a \in \Delta^\iota | \forall b.  (a, b) \in R^\iota \rightarrow b \in C^\iota \}$ & $AL$ \\ \hline
Restri��o existencial & $\exists R.C$ & $\{ a \in \Delta^\iota | \exists b.  (a, b) \in R^\iota \land b \in C^\iota\}$ & $\epsilon$ \\ \hline

Restri��o & $\geq n R$ & $\{ a \in \Delta^\iota || \{ b \in \Delta^\iota | (a, b) \in R^\iota \} | \geq n\}$ & \\ 
num�rica & $\leq n R$ & $\{ a \in \Delta^\iota || \{ b \in \Delta^\iota | (a, b) \in R^\iota \} | \leq n\}$ & $N$ \\ 
n�o qualificada & $= n R$ & $\{ a \in \Delta^\iota || \{ b \in \Delta^\iota | (a, b) \in R^\iota \} | = n\}$ & \\ \hline

Restri��o & $\geq n R.C$ & $\{ a \in \Delta^\iota || \{ b \in \Delta^\iota | (a, b) \in R^\iota \land b \in C^\iota \} | \geq n\}$ & \\ 
num�rica & $\leq n R.C$ & $\{ a \in \Delta^\iota || \{ b \in \Delta^\iota | (a, b) \in R^\iota  \land b \in C^\iota \} | \leq n\}$ & $Q$ \\ 
qualificada & $= n R.C$ & $\{ a \in \Delta^\iota || \{ b \in \Delta^\iota | (a, b) \in R^\iota  \land b \in C^\iota \} | = n\}$ & \\ \hline

\end{tabular}
\end{center}
\caption{Sintaxe e sem�ntica de alguns contructos de l�gica de descri��o com anota��o de expressividade}
\end{table}


\section{OWL: Web Ontology Language}

A \textit{Web Ontology Language}, OWL, foi escolhida pela w3c \footnote{sitio oficial: http://w3.org}, grupo que regula os padr�es na Web, como a linguagem de descri��o de ontologias para a Web Sem�ntica \cite{Heflin:2004}. Alguns do requisitos que ela atendeu foram \cite{Antoniou:2008}: i) sintaxe bem definida; Como o objetivo da Web Sem�ntica � tornar os documendos da Web mais f�ceis de serem processados por m�quinas, este � um requisito b�sico. ii) sem�ntica formal; Descrever a base de conhecimento de forma l�gicamente precisa � fundamental para fazer infer�ncias como dedu��o de conceitos, checagem de consist�ncia na base de conhecimento e instancia��o de indiv�duos a uma classe. iii) suporte a racioc�nio; Uma vez que a linguagem possui uma sem�ntica formal, atividades de racioc�nio podem ser realizadas. iv) expressividade; alguns dom�nios precisam de construtos mais elaborados para que possam ser descritos. Quanto maior a expressividade da linguagem, naturalmente fica mais f�cil de descrever um dom�nio, apesar de aumentar a complexidade e tempo de processamento.

Entre os requisitos citados no paragrafo anterior est�o expressividade e suporte a racioc�nio. Apesar de ambos poderem estar na linguagem, s�o antag�nicos, quanto maior for a expressividade da linguagem, mais complexas e demoradas ser�o as atividades de racioc�nio sobre a linguagem. Para criar fronteiras nesse conflito entre expressividade e complexidade de racioc�nio, a w3c criou tr�s vers�es de OWL: OWL Full, OWL DL e OWL Lite.

\subsection{OWL Full}

A Web Ontology Language em sua vers�o mais expressiva, usando todas as primitivas da linguagem, � chamada de OWL Full. Essa combina��o inclui, por exemplo, aplicar uma restri��o de cardinalidade na classe que cont�m todas as outras classes, limitando a quantidade de classes que a ontologia pode ter.

OWL Full � completamente compat�vel com RDF, tanto sintaticamente, quanto em sua sem�ntica. A desvantagem de OWL Full � que ela � t�o poderosa que � indecid�vel em rela��o �s atividades de racioc�nio.

\subsection{OWL DL}

OWL DL (DL � a sigla para \textit{Description Logic}, L�gica de Descri��o em portugu�s) � a fam�lia de OWL que corresponde � l�gica de Descri��o. A sua grande vantagem � que ela � decid�vel, dando a possibilidade de realiza��o de atividades de racioc�nio de forma mais eficiente. A desvantagem de OWL DL � que ela perde a compatibilidade com RDF, qualquer documento em OWL DL pode ser descrito como um documento em RDF, mas o contr�rio n�o � verdade.

\subsection{OWL Lite}

OWL Lite � uma fam�lia de OWL que � mais limitada do que OWL Full e OWL DL. Ela n�o d� suporte a, por exemplo, disjun��o entre classes, uni�o e complemento. A grande vantagem dessa linguagem � uma maior facilidade para o desenvolvimento de ferramentas, e a sua desvantagem � a perda de expressividade.

\vspace{2 cm}

\noindent
sdfsdf


\section{Normaliza��o para o m�todo das conex�es}
% !TEX encoding = ISO-8859-1
\chapter{Normaliza��o para o m�todo das conex�es}
\label{ch:normalizacaoparaometododeconexoes}

O m�todo das conex�es proposto por W. Bibel \cite{bibel:1982} � um m�todo para prova autom�tica de teoremas descritos em l�gica d primeira ordem ou em l�gica proposicional \cite{bibel:1993}. Um dos trabalhos recentes de Freitas et al \cite{Freitas:2010} foi a exten��o desse m�todo para l�gica de descri��o $ALC$.

O artigo intitulado \textit{A Connection Method for Reasoning with the Description Logic ALC} \cite{Freitas:2010} prop�e algoritmos tanto para o m�todo, quanto para a normaliza��o que precisa ser feita na base de conhecimento para que seja poss�vel a representa��o necess�ria para o m�todo das conex�es usando apenas uma matriz. 

O objetivo deste trabalho � implementar algum algoritmo de normaliza��o para o m�todo das conex�es como os citados no texto de Freitas et al. Dois algoritmos foram propostos com esse objetivo \cite{Freitas:2010}; o primeiro utiliza-se de uma tabela com nove regras que devem ser aplicadas � base de conhecimento a fim de obter a forma normal positiva. O segundo, intitulado "\textit{A more complex and efficient normalization}" n�o cria novos s�mbolos durante a sua execu��o, fazendo-o mais eficiente que o primeiro em rela��o ao uso de mem�ria.

No cronograma deste trabalho estava prevista a implementa��o desse segundo algoritmo, por�m, ao decorrer do desenvolvimento, o orientando prop�s um terceiro algoritmo que � ainda mais eficiente em rela��o ao uso de mem�ria, mas com o efeito colateral de precisar adicionar um s�mbolo a mais na base de conhecimento no pior caso. O restante desse cap�tulo se dedicar� a dar defini��es para o entendimento dos �ltimos dois algoritmos comentados acima e tamb�m descrev� as suas implementa��es.

\section{Tradu��o de ontologias $ALC$ para forma normal disjuntiva}

Para que o leitor consiga entender melhor os algoritmos de tradu��o, alguns conceitos precisam ser fixados. 

M�todos diretos com o m�todo das conex�es s�o formulados para provar que uma f�rmula ou um conjunto de f�rmulas � um teorema, sse cada interpreta��o gerada � uma tautologia. Tautologias normalmente tomam a forma $L \lor \neg L$, nesse caso, a f�rmula precisa estar na Forma Normal Disjuntiva (DNF).

\begin{definicao}[Forma Normal Disjuntiva, cl�usula]
 Uma f�rmula em DNF � uma conjun��o de disjun��es. Ou seja, tomam a forma:

\begin{center}
$\bigcup \limits_{i=1}^{n} C_i$ , ou, $C_1 \lor ... \lor C_n$.
\end{center}

onde cada $C_i$ � uma \textit{cl�usula}. Uma cl�usula � uma conjun��o de literais. Ou seja, tomam a forma:

\begin{center}
$\bigcap \limits_{j=1}^m L_{i, j}$ , ou, $L_{i, 1} \land ... \land L_{i, m}$ , tamb�m representado por $\{ L_{i, 1}, ..., L_{i, m} \}$
\end{center}

onde cada $L_{i, j}$ � um literal, resultando na f�rmula:

\begin{center}
$\bigcap \limits_{i=1}^n \bigcup \limits_{j=1}^m L_{i, j}$ , ou, $(L_{1,1} \land ... \land L_{1,m}) \lor (L_{n, 1} \land ... \land L_{n, m})$
\end{center}

podendo ser chamada tamb�m de forma causal disjuntiva, representada por: 

\begin{center}
$\{\{ (L_{1, 1}, ..., L_{1, m}\}, ..., \{L_{n, 1}, ..., L_{n, m} \}\}$
\end{center}

\end{definicao}

A defini��o acima � a defini��o herdada da l�gica de primeira ordem, para ser v�lida tamb�m para a l�gica de descri��o o conceito de conjun��es e disjun��es deve ser estendido.

\begin{definicao}[Conjun��o ALC]
Uma conjun��o ALC � um literal $L$, uma conjun��o $(E_0 \land , ..., \land E_n)$, ou uma restri��o existencial $\exists x.E$, onde $E$ � uma express�o qualquer em l�gica de descri��o.
\end{definicao}

\begin{definicao}[Disjun��o ALC]
Uma disjun��o ALC � um literal $L$, uma disjun��o $(E_0 \lor , ..., \lor E_n)$, ou uma restri��o de valor $\forall x.E$, onde $E$ � uma express�o qualquer em l�gica de descri��o.
\end{definicao}

\begin{definicao}[Conjun��o ALC pura, Conjun��o ALC n�o pura]
Uma conjun��o ALC pura � uma conjun��o ALC que na forma normal negada n�o cont�m restri��es de valor ($\forall x.E$) e tamb�m n�o cont�m disjun��es $(E \lor , ..., \lor E)$. O conjunto de conjun��es ALC puras � representado por $\hat{C}$. Uma conjun��o ALC n�o pura � uma conjun��o ALC que n�o � pura.
\end{definicao}

\begin{definicao}[Disjun��o ALC pura, Disjun��o ALC n�o pura]
Uma disjun��o ALC pura � uma disjun��o ALC que na forma normal negada n�o cont�m restri��es existenciais ($\exists x.E$) e tamb�m n�o cont�m conjun��es $(E \land , ..., \land E)$. O conjunto de disjun��es ALC puras � representado por $\check{D}$. Uma disjun��o ALC n�o pura � uma disjun��o ALC que n�o � pura.
\end{definicao}

\begin{definicao}[Impureza em uma express�o n�o pura]
Impureza em express�es ALC n�o puras s�o conjun��es em disjun��es n�o puras ou disjun��es em conjun��es n�o puras. O conjunto de impurezas � chamado de \text{conjunto de impurezas ALC} e � representado por $I$.
\end{definicao}

\begin{definicao}[Forma Normal Positiva]
Um axioma ALC est� na Forma Normal Positiva sse ele est� em uma das seguintes formas: 

\begin{center}
i) $A \sqsubseteq \hat{C}$ \\
ii) $\check{D} \sqsubseteq A$ \\
iii) $\hat{C} \sqsubseteq \check{D}$ \\ 
\end{center}

onde A � um conceito at�mico,  $\hat{C}$ � uma conjun��o ALC pura, $\check{D}$ � uma disjun��o ALC pura.

\end{definicao}

O m�todo das conex�es utiliza matrizes para realizar provas de teoremas. No in�cio deste trabalho, ainda n�o era poss�vel de fazer as provas com matrizes aninhadas, ou seja, havia sempre a necessidade de normalizar a base de conhecimento na forma normal positiva (defini��o 7). No entanto, Jens Otten em um trabalho entitulado \textit{A Non-clausal Connection Calculus} \cite{Otten:2011} mostrou como aplicar o m�todo das conex�es sem o passo da normaliza��o. Apesar da recente evoluc�o do m�todo, para o objetivo deste trabalho, ainda � necess�rio fazer a normaliza��o, j� que para l�gica de descri��o, o m�todo ainda n�o foi modificado para usar matrizes aninhadas.

A pr�xima se��o desta monografia descreve o algoritmo proposto pelo autor para a normaliza��o para a forma normal positiva.


\subsection{Algoritmo Proposto}

Esta se��o da monografia descreve a implementa��o realizada neste trabalho.

O m�todo das conex�es � um m�todo direto, ou seja, uma consulta � base de conhecimento toma verifica se uma f�rmla � uma tautologia e toma a forma $KB \rightarrow \alpha$, onde $\alpha$ � um axioma e $KB$ (\textit{Knowledge base}) � da forma $\bigcap \limits_{i=1}^{n} A_i$, onde $A_i$ tamb�m � um axioma. Expandindo a f�rmula $\neg KB \lor \alpha$, temos:

$\neg \bigcap \limits_{i=1}^{n} A_i \lor \alpha$ [ou, $\neg (A_1 \land ... \land A_n) \lor \alpha$], que pode ser transformada para:

$\bigcup \limits_{i=1}^{n} \neg A_i \lor \alpha$ [ou, $\neg A_1 \lor ... \lor \neg A_n \lor \alpha$]

%, e $\alpha$ ainda � transformada para $\neg \neg\alpha$ para que as regras de normaliza��o tamb�m se apliquem a ela, ficando:

%$\bigcup \limits_{i=1}^{n} \neg A_i \lor \neg \neg \alpha$ [ou, em uma forma por extenso, $\neg A_1 \lor ... \lor \neg A_n \lor \neg \neg \alpha$]

Cada $\neg A_i$ ou $\alpha$ precisam estar na forma normal positiva, ou seja, precisam estar em uma das formas: i) $A \sqsubseteq \hat{C}$, ii) $\check{D} \sqsubseteq A$, ou, iii) $\hat{C} \sqsubseteq \check{D}$, onde $A$ � um conceito, $\hat{C}$ � uma conjun��o pura e $\check{D}$ � uma disjun��o pura.

O primeiro passo do algoritmo � a separa��o de axiomas de equival�ncia de express�es e inclus�es. Os axiomas de equival�ncia ser�o substitu�dos por dois axiomas de inclus�o, e.g., $(A \equiv B \rightarrow A \sqsubseteq B \land B \sqsubseteq A)$.

\begin{algorithm}
\caption{In�cio do processo de normaliza��o para toda a base de conhecimento}
\begin{algorithmic}
\FORALL [A, B s�o express�es, $S_{EQ}$ � o conjunto dos axiomas de equival�ncia]{$A \equiv B \in S_{EQ}$}
\STATE Normalize-Axiom ($A \sqsubseteq B$)
\STATE Normalize-Axiom ($B \sqsubseteq A$)
\ENDFOR
\FORALL [A, B s�o express�es, $S_I$ � o conjunto dos axiomas de inclus�o]{$A \sqsubseteq B \in S_I$}
\STATE Normalize-Axiom ($A \sqsubseteq B$)
\ENDFOR
\end{algorithmic}
\end{algorithm}

A chamada ao m�todo \textit{Normalize-Axiom()} remove as impuresas do lado esquerdo e do lado direito de cada axioma.

\begin{algorithm}
\caption{Normalize-Axiom ($A \sqsubseteq B)$)}
\begin{algorithmic}
\STATE $O = \{O - (A \sqsubseteq B) \} $
\STATE $pure\_left = $ purify $(A)$
\STATE $pure\_right = $ purify $(B)$
\IF [$S_{PNF}$ � o conjunto das f�rmulas na forma normal positiva]{$(pure\_left \sqsubseteq pure\_right) \in S_{PNF}$}
\STATE $O = \{O \sqcup (pure\_left \sqsubseteq pure\_right)\} $
\ELSE
\STATE $O = \{O \sqcup (pure\_left \sqsubseteq N)\} $
\STATE $O = \{O \sqcup (N \sqsubseteq pure\_right)\} $
\ENDIF
\end{algorithmic}
\end{algorithm}

O m�todo $purify (A)$ indentifica se $A$ (express�o em DL) � uma conjun��o, disjun��o ou conceito. Caso $A$ seja uma conjun��o n�o pura, as impuresas s�o removidas, caso contr�rio, $A$ permanece com a estrutura original. De forma an�loga, caso $A$ seja uma disjun��o n�o pura, as impuresas ser�o removidas e $A$ passar� a ser pura. E o �ltimo caso, caso $A$ seja um conceito, ele n�o ser� modificado. Ao remover uma impuresa, um axioma � adicionado � base com a express�o impura. O m�todo \textit{Normalize-Axiom()} � ent�o chamado com esse novo axioma adicionado.

O $if$ do m�todo \textit{Normalize-Axiom ()} verifica se o axioma $A \sqsubseteq B$ est� na forma normal positiva, caso n�o esteja, as express�es s�o separadas em dois axiomas para respeitar a tabela 3.1.

\begin{table}
\begin{center}
\begin{tabular}{ | c | c | } \hline
express�o & regra \\ \hline
$\hat{C_1} \sqsubseteq \hat{C_2}$ & $(\hat{C_1} \sqsubseteq A' ) \land (A' \sqsubseteq \hat{C_2})$  \\ \hline
$\check{D_1} \sqsubseteq \check{D_2}$ & $(\check{D_1} \sqsubseteq A' ) \land (A' \sqsubseteq \check{D_2})$ \\ \hline
$\check{D} \sqsubseteq \hat{C}$ & $(\check{D} \sqsubseteq A' ) \land (A' \sqsubseteq \hat{C})$ \\ \hline

\end{tabular}
\caption{Tabela de regras a serem aplicadas ap�s o m�todo purify}
\end{center}
\end{table}

\subsection{Algoritmo Original}

Esta se��o cont�m o algoritmo escrito por Freitas et al \cite{Freitas:2010} para a normaliza��o de bases de conhecimento em l�gica de descri��o $ALC$. A figura 3.1 descreve o m�todo \textit{Normalize-Ontology ()}, que chama os m�todos \textit{Normaize-LHS ()} e \textit{Normalize-RHS ()} caso o lado esquerdo e direito, respectivamente,  n�o estejam normalizados para cada axioma da ontologia.

\begin{figure}
\includegraphics[scale=0.7]{../imagens/normalize.png}
\caption{M�todo \textit{Normalize-Ontology ()}}
\label{fig:normalizeontology}
\end{figure}

Os m�todos \textit{Normalize-LHS ()} e \textit{Normalize-RHS} s�o duais, ou seja, possuem a mesma l�gica mas com o sentido invertido. A figura 3.2 mostra o pseudoc�digo do m�todo \textit{Normalize-LHS()} e a figura 3.3 mostra o pseudoc�digo do m�todo \textit{Normalize-RHS}.

 \begin{figure}
\includegraphics[scale=0.52]{../imagens/normalizelhs.png}
\caption{M�todo \textit{Normalize-LHS ()}}
\label{fig:normalizelhs}
\end{figure}

 \begin{figure}
\includegraphics[scale=0.52]{../imagens/normalizerhs.png}
\caption{M�todo \textit{Normalize-RHS ()}}
\label{fig:normalizerhs}
\end{figure}

Note que o primeiro algoritmo da se��o 3.1.1 e a figura 3.1 s�o praticamente o mesmo, por�m, o primeiro diferencia dentro do m�todo os axiomas de equival�ncia e inclus�o. No artigo de Freitas et al \cite{Freitas:2010} � citado que isso deve ser feito, mas n�o explicita no algoritmo.

Ao comparar o segundo algoritmo da se��o 3.1.1 com as figuras 3.2 e 3.3 pode ser visto de forma clara que o algoritmo de normaliza��o proposto ficou muito mais simples que o original.


%\section{Conclus�o e trabalhos futuros}
% !TEX encoding = ISO-8859-1
%\chapter{Conclus�o e trabalhos futuros}
%\label{ch:conclusaoetrabalhosfuturos}

\section{Conclus�o}
Os resultados de boa performance do leanCoP para l�gica de primeira ordem d�o ind�cios que o m�todo das conex�es pode ganhar espa�o entre os m�todos de prova para l�gica de descri��o. Como a w3c \footnote{site: http://w3.org} ainda n�o definiu uma tecnologia padr�o para a camada de l�gica e prova da Web Sem�ntica, trabalhos como o que este est� inserido s�o de import�ncia estrat�gica para a Web, eles desenvolvem solu��es que poder�o ser adotados em larga escala pelo mundo.

O leanCoP foi envolvido neste trabalho para realizar atividades de racioc�nio, como subsun��o e equival�ncia, ap�s a normaliza��o da base de conhecimento. O leanCoP � escrito em Prolog e as APIs de manipula��o de ontologias OWL s�o escritas em Java. Para fazer o leanCoP usar como base de conhecimento as ontologias OWL normalizadas, foi utilizado o formato TPTP como linguagem intermedi�ria. Por�m, para fazer as atividades de racioc�nio de forma autom�tica, um n�mero exponencial de arquivos deveriam ser gerados em TPTP para serem usados com o leanCoP. Al�m disso, um trabalho de \textit{parsing} desses arquivos deveria ser realizado para adicionar cada consulta � base de conhecimento de cada arquivo, o que se mostrou fora do escopo do projeto. O leanCoP foi utilizado ent�o para fazer simples checagem de consist�ncia na base de conhecimento, que traz como efeito colateral a valida��o da corretude do algoritmo de normaliza��o.

Na proposta inicial deste trabalho estava prevista a implementa��o do algoritmo de normaliza��o de Freitas et al \cite{Freitas:2010}, por�m, apesar do algoritmo n�o incluir novos s�mbolos � base de conhecimento, n�o � f�cil de ser entendido. A se��o 3.1.1 mostra em pseudo-c�digo a implementa��o que foi feita neste trabalho. O algoritmo foi produzido devido a uma provoca��o de simplificar o algoritmo de Freitas et al \cite{Freitas:2010}. O objetivo dos algoritmos � o mesmo, traduzir os axiomas de uma ontologia ALC para a forma normal positiva, mas o algoritmo da se��o 3.1.1 al�m de ser mais simples de ser implementado, faz a matriz gerada ap�s a normaliza��o consumir menos mem�ria, o que foi uma grande contribui��o. A redu��o do uso de mem�ria vai impactar no tempo de execu��o do m�todo das conex�es para l�gica de descri��o, j� que a busca pelos caminhos na matriz vai ser reduzido.

\section{Trabalhos futuros}

Este trabalho n�o contempla todos os constructos de OWL, nem sequer de OWL Lite, j� que � limitada � familia ALC. Trabalhos futuros ser�o para estender o algoritmo de normaliza��o para incluir restri��es com cardinalidade, dom�nio e contradom�nio de propriedades, disjun��o entre classes e assim por diante.

Este trabalho � apenas um dos m�dulos necess�rios para a implementa��o de um raciocinador escrito em java que use o m�todo das conex�es. O algoritmo em si que procura pelas conex�es, ou caminhos, ainda n�o foi implementado.

E, por fim, quando o m�todo das conex�es estiver formalizado para uma fam�lia de DL que seja equivalente a uma fam�lia de OWL e a sua implementa��o estiver finalizada, poder� haver um trabalho para integrar o raciocinador a editores de ontologias existentes no mercado, como o Prot�g�.



\bibliographystyle{sbc}
\bibliography{../referencias/referencias}

\end{document}
