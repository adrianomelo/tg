% !TEX encoding = ISO-8859-1
\documentclass[bsc]{ufpethesis}

%\university{<NOME DA UNIVERSIDADE>}
%\address{<CIDADE DA IES>}
\institute{Centro de Inform�tica}
%\department{Centro de Inform�tica}
\program{Gradua��o em Ci�ncia da Computa��o}
\majorfield{Ci�ncia da Computa��o}
\title{Normaliza��o de ontologias ALC para o uso com o LeanCop}
\date{11 de julho de 2011}
\author{Adriano Silva Tavares de Melo}
\adviser{Frederico Freitas}
%\coadviser{NOME DO(DA) CO-ORIENTADOR(A)}

\begin{document}

%% Parte pr�-textual
\frontmatter
% Folha de rosto
\frontpage
% Portada (apresenta��o)
\presentationpage
% Dedicat�ria
\begin{dedicatory}
Eu dedico esse trabalho a minha m�e, namorada, familiares e amigos.
\end{dedicatory}
% Agradecimentos
\acknowledgements
% !TEX encoding = ISO-8859-1
Gostaria de agradecer � minha m�e, que sempre acreditou em mim, que me deu condi��es de estudar, que abdicou de v�rias coisas ao decorrer de sua vida para dar prioridade aos filhos, que nos incentivou a crescer, que nos ensinou valores, que, em fim, nos amou incondicionalmente.

Agrade�o a todos da minha fam�lia que me apoiaram a estudar e a trilhar caminhos que me deixaram longe, em especial ao meu irm�o, madrinha e av�, que sempre me falam o quanto s�o orgulhosos da minha decis�o de sair da minha cidade para evoluir profissionalmente. Obrigado!

Agrade�o � minha namorada, que tornou essa jornada longe dos amigos e fam�lia mais agrad�vel e feliz.

Agrade�o a Fred, que sempre que precisei se mostrou disposto a ajudar e que me ensinou muito nos �ltimos anos.

Agrade�o a todos que me ajudaram profissionalmente e academicamente, eu n�o teria conseguido chegar at� aqui sem essa ajuda. Obrigado!

%Gostaria de agradecer primeiro a minha m�e que me deu condi��es de sair de minha cidade natal para fazer uma gradua��o, m�e, amo-te!

\begin{epigraph}[<NOTA>]{<AUTOR>}
<DIGITE AQUI A CITA��O>
\end{epigraph}

\resumo
A World Wide Web revolucionou a comunica��o entre as pessoas em todo o mundo. Com ela, o custo de enviar uma informa��o de qualquer lugar do mundo para qualquer lugar do mundo foi reduzida a praticamente a zero. Essa facilidade de acesso a informa��o global fez com que ela crescesse de forma exponencial. 
\begin{keywords}
m�todo de conex�es, l�gica de descri��o, normaliza��o
\end{keywords}

\abstract
\begin{keywords}
<DIGITE AS PALAVRAS-CHAVE AQUI>
\end{keywords}

\tableofcontents
\listoffigures
\listoftables

%% Parte textual
\mainmatter

% !TEX encoding = ISO-8859-1
%\chapter{Introdu��o}
%\label{ch:introducao}

%\section{Contextualiza��o}
A World Wide Web � uma das tecnologias mais revolucion�rias que o homem j� inventou. Ela mudou em escala global a forma com que pessoas e empresas trocam informa��es, contribuindo para que o conhecimento se tornasse mais universal e que limites f�sicos e ling��sticos fossem cada vez mais minimizados.

%Tim Berners-Lee, o inventor da Web, tinha preten��es bem definidas ao elaborar essa infra-estrutura, seu principal objetivo era melhorar a forma com que o conhecimento era mantido dentro de uma empresa, e nessa concep��o, o que importaria era a visualiza��o dos dados por humanos, e n�o pelas m�quinas. Hoje, a preocupa��o � outra, como estruturar melhor as informa��es que s�o abundantes na Web para que sejam compreendias pelas m�quinas?

A web como conhecemos hoje nasceu de uma proposta feita por Tim Berners-Lee � empresa CERN em 1989 \cite{BernersLee:1989}. O problema enfrentado pela empresa na �poca era a perca de informa��es internas por falta de documenta��o ou pela sa�da de algum funcion�rio. A solu��o proposta por Berners-Lee foi fazer uma rede de documentos interligados por hyperlinks em que cada setor da empresa poderia adicionar novos documentos. 

A estrutura b�sica que Berners-Lee montou a 22 anos evoluiu a passos largos em rela��o � escalabilidade e padroniza��o de protocolos e linguagens, tendo hoje cerca de 2 bilh�es de usu�rios, mais de 30\% da popula��o do planeta. 

Apesar do avan�o das infra-estruturas e servi�os para a Web, ainda h� muito o que evoluir. Uma das propostas de mudan�as � prover uma maior expressividade da linguagem que descreve os documentos na Web \cite{Heflin:2004}. Hoje, esses documentos n�o possuem um significado que possa ser extra�do de forma concisa, apresentam ambig�idade, misturam os dados com elementos visuais e muitas vezes n�o podem ser indexados por engenhos de busca.

\section{Web Sem�ntica}

\begin{quote}
\textit{"I have a dream for the Web [in which computers] become capable of analyzing all the data on the Web, the content, links, and transactions between people and computers. A \textbf{Semantic Web} which should make this possible has yet to emerge, but when it does, the day-to-day mechanisms of trade, bureaucracy and our daily lives will be handled by machines talking to machines. The \textbf{intelligent agents} people have touted for ages will finally materialize." Tim Berners-Lee}
\end{quote}

\begin{quote}
Tradu��o literal: \textit{"Eu tenho um sonho para a Web [em que os computadores] tornam-se capazes de analisar todos os dados na Web, o conte�do, links, e as transa��es entre pessoas e computadores. A \textbf{Web Sem�ntica} que deve tornar isso poss�vel ainda est� para surgir, mas quando isso acontecer, os mecanismos dia-a-dia da burocracia do com�rcio e nossas vidas di�rias ser�o tratados por m�quinas falando com m�quinas. Os \textbf{agentes inteligentes} que as pessoas t�m falado por anos v�o finalmente se concretizar." Tim Berners-Lee}
\end{quote}

A Web Sem�ntica citada no texto de Berners-Lee acima � uma iniciativa de pesquisadores da �rea de intelig�ncia artificial e ling��stica computacional que estudam como adequar a Web de hoje a uma infra-estrutura que a tornar� mais acess�vel �s m�quinas. Essa nova roupagem que os pesquisadores querem dar � Web permitir� que servi�os mais sofisticados possam ser constru�dos, como os que ser�o descritos a seguir.

\subsection{Aplica��es}

\subsubsection{Gerenciamento de Conhecimento}
%\textbf{Gerenciamento de Conhecimento}

Gerenciamento de conhecimento est� relacionado � aquisi��o, acesso e manuten��o de conhecimento dentro de uma empresa ou organiza��o. Essa atividade se tornou e est� se estabelecendo como uma necessidade b�sica em grandes empresas visto que o conhecimento que � gerado internamente agrega valor, pode se tornar um diferencial competitivo e tamb�m pode aumentar a produtividade de seus colaboradores. Com o uso de tecnologias criadas para a Web Sem�ntica, solu��es para G.C. podem melhorar em v�rios aspectos, entre eles:

\begin{itemize}
\item Organiza��o do conhecimento existente a partir de seu significado;
\item Gera��o de novas informa��es de forma autom�tica;
\item Checagem de inconsist�ncias sem�nticas em documentos;
\item Substitui��o de consultas baseadas em palavras-chave por perguntas em linguagem natural; 
\end{itemize}

\subsubsection{Com�rcio eletr�nico \textit{Business to Consumer} (B2C)}
%\textbf{Com�rcio eletr�nico Business to Consumer (B2C)}

O com�rcio eletr�nico entre vendedores e consumidores � um dos modelos de neg�cio na Internet que melhor se estabeleceu, sites como amazon \footnote{site: amazon.com}, americanas \footnote{site: americanas.com} e mercado livre \footnote{site: mercadolivre.com.br} possuem p�blico fiel e que os visitam por v�rios objetivos. � muito comum para a gera��o que cresceu imersa na Web entrar em sites de compra como esses a procura do melhor pre�o antes de decidir fazer uma compra. Muitas vezes o produto n�o � adquirido em uma loja virtual, mas a pesquisa inicial de pre�os � que muitas vezes determina a escolha do produto. Observando esse comportamento, sites como o buscap� \footnote{site: buscape.com.br} fazem o trabalho de indicar qual � a loja que est� com o melhor pre�o. 

A Web Sem�ntica pode ajudar nesse cen�rio provendo interfaces de consulta mais completas aos sites que fazem compara��o de pre�os, por�m, com muito mais detalhes t�cnicos sobre o produto. Supondo que cada produto tem, por exemplo, uma ontologia que o descreve em detalhes (provida pelo fabricante ou por sites de review de produtos), o consumidor poder� fazer compara��es muito mais detalhadas, ajudando-o a encontrar o produto que vai suprir a sua necessidade.

\subsubsection{Com�rcio eletr�nico \textit{Business to Business} (B2B) e agentes pessoais}
%\textbf{Com�rcio eletr�nico Business to Business (B2B) e agentes pessoais}

A maioria das pessoas que compram servi�os ou produtos na Web s� conhecem o com�rcio eletr�nico do tipo B2C, mas existem tecnologias para com�rcio do tipo B2B, Business to Business, agentes computacionais de empresas que se comunicam para fechar acordos e otimizar o ciclo de neg�cios que muitas vezes j� podem ser previstos e modelados.

Com a populariza��o da Web Sem�ntica e a introdu��o de agentes pessoais e agentes que representam neg�cios, eles poder�o se comunicar de forma mais natural e aplica��es para otimizar tarefas do dia-a-dia que poder�o ser produzidos. Por exemplo, um m�dico que possua um agente pessoal que negocie a sua agenda com os agentes pessoais de seus clientes pode ser utilizado para remarcar seus atendimentos em caso de uma viagem ou imprevisto, agindo como uma secret�ria virtual.

\subsection {Tecnologias}

Aplica��es como as citadas acima j� existem, mas o trabalho de engenharia para conseguir bons resultados � alto devido �s tecnologias que s�o adotadas hoje. Vamos usar o case do \textit{site} de compara��o de pre�os BuscaP� na pr�xima se��o da monografia. 

\subsubsection{Metadados Expl�citos}

A primeira tarefa de engenharia de um agente coletor de pre�o, como os do buscaP�, � faz�-lo visitar v�rios sites de compras todo dia � procura de modifica��es nas listas de produtos para saber quais est�o dispon�veis naquela loja. � feito ent�o o \textit{parsing} do HTML de cada site de compras � procura das informa��es de pre�o, descri��o, avalia��o e detalhes de cada produto. A limita��o dessa abordagem � que sempre que um dos sites de compra mudar o \textit{layout} (estrutura do HTML), um novo script de \textit{parsing} dever� ser escrito. A grande demanda t�cnica de uma aplica��o como essa � a escrita de agentes muito especializados para atingir bons resultados.

Na figura 1.1 est� parte do c�digo em HTML de uma p�gina de produtos da americanas.com. As informa��es do produto est�o cercados apenas de c�digo para a renderiza��o dessa p�gina pelo navegador. Ou seja, a �nica preocupa��o dos engenheiros da americanas.com foi a leitura por humanos da informa��o do produto. Uma aplica��o que deseje usar as informa��es dos produtos da loja vai ter que fazer um agente especializado no \textit{parsing} desse c�digo.

\begin{figure}
\includegraphics[scale=0.55]{../imagens/americanas.png}
\caption{C�digo HTML de uma p�gina da americanas.com}
\label{fig:americanas}
\end{figure}

Motores de busca que tamb�m se baseiam em \textit{parsing} de p�ginas para extrair informa��es da Web dificilmente saber�o, por exemplo, qual � o pre�o de um produto nesse site da americanas.com, j� que h� v�rias informa��es de pre�o na p�gina e o \textit{parsing} que � feito n�o � otimizado para sites espec�ficos.

A conseq��ncia para o usu�rio final � que ele ter� que usar um site espec�fico como o buscap� ou ter� que fazer buscas a um engenho de busca por palavras-chave para achar os sites de compra que possuam um produto e em uma segunda etapa, fazer a an�lise de pre�os manualmente.

A abordagem da Web Sem�ntica para resolver problemas como esse n�o � fazer agentes especializados (como os do buscap�), e sim, anotar metadados sem�nticos dos documentos dispon�veis na Web. O exemplo dado anteriormente seria escrito na figura 1.2.

\begin{figure}
\includegraphics[scale=0.55]{../imagens/americanas-xml.png}
\caption{Exemplo de c�digo com sem�ntica para um produto}
\label{fig:americanasxml}
\end{figure}

\subsubsection{Ontologias}

O termo ontologia vem da filosofia, nesse contexto, � um ramo da filosofia que se dedica a estudar a natureza da exist�ncia, concentra-se em identificar e descrever o qu existe no universo. Em computa��o, uma ontologia � um artefado para descrever um dom�nio. Consiste em uma lista finita de termos e rela��es entre eles. Os termos denotam conceitos importantes de um dom�nio \cite{Antoniou:2008}.

Grande parte dos trabalhos referentes � Web Semantica est�o ligados a ontologias, inclusive este. As linguagem de descri��o de ontologias mais import�ntes para a Web s�o:

\begin{itemize}
\item XML: usado para dirigir a sintaxe de documentos estruturados. N�o imp�e restri��es sem�nticas no conte�do do documento;
\item XML Schema: linguagem para impor restri��es na estrutura dos documentos XML;
\item RDF: modelo de dados para recursos (objetos) e rela��es entre eles. As restri��es sem�nticas s�o fixas e podem ser representados a partir da sintaxe do XML;
\item RDF Schema: descreve as propriedades e classes dos objetos RDF;
\item OWL: linguagem rica para modelagem de classes, propriedades, rela��es entre classes (e.g. disjun��o), restri��es de cardinalidade, caracter�sticas de propriedades (e.g. simetria). Mais detalhes sobre a OWL ser�o dados no cap�tulo 2 dessa monografia.
\end{itemize}

\subsubsection{L�gica}

L�gica � a disciplina que estuda os princ�pios do racioc�nio. Ela prov� linguagens formais para expressar conhecimento, a sem�ntica formal para a interpreta��o de senten�as sem precisar realizar opera��es sobre a base de conhecimento e a transforma��o de conhecimento impl�cito em conhecimento expl�cito, atrav�s de dedu��es a partir da base de conhecimento \cite{Antoniou:2008}.

L�gica � mais geral que ontologias, ela pode ser usada por agentes inteligentes para tomada de decis�es e escolha de a��es. Por exemplo, um agente de B2C pode dar um desconto a um cliente baseado na seguinte regra:

\[
\forall x \forall y, cliente(x) \land produto(y) \land clienteFiel(x) \rightarrow desconto(x, y, 5\%)
\]

Onde \textit{cliente(x)} indica que x � um cliente/consumidor, \textit{produto(y)} indica que y � um produto de uma loja, \textit{clienteFiel(x)} indica que x � um cliente fiel da loja e \textit{desconto(x, y, 5\%)} indica que o cliente x ter� um desconto de 5\% no produto y.

\subsubsection{Agentes}

Um agente � tudo o que pode ser considerado capaz de perceber seu ambiente por meio de sensores e de agir sobre esse ambiente por meio de atuadores \cite{AIMA}. Agentes l�gicos s�o aqueles que executam a��es atrav�s de uma base de conhecimento e possuem um requisito fundamental, quando ele formula uma pergunta para a base de conhecimento, a resposta deve seguir o que j� foi informado anteriormente.

Agentes para a Web Sem�ntica utilizam as tr�s tecnologias que j� foram descritas:

\begin{itemize}
\item Metadados ser�o usados para identificar e extrair informa��es da Web;
\item Ontologias ser�o usadas para dar assist�ncia �s consultas realizadas � Web, interpretar informa��es recuperadas e para comunica��o com outros agentes;
\item L�gica ser� usada para processar informa��es recuperadas, chegar a conclus�es e tomar decis�es;
\end{itemize}

\section{Organiza��o da Monografia}

Esta monografia est� dividida em cinco cap�tulos. No Cap�tulo 1, � apresentada uma vis�o geral sobre a Web Sem�ntica, exemplificando com aplica��es e citando as tecnologias que est�o sendo usadas. No Cap�tulo 2, s�o apresentados conceitos referentes a L�gica de Descri��o e a sua liga��o com a linguagem de descri��o de ontologias OWL. No Cap�tulo 3 s�o descritos os algoritmos para normaliza��o de ontologias para a Forma Normal Positiva. No Cap�tulo 4 o LeanCop � apresentado e � mostrada a valida��o do trabalho realizado. O Cap�tulo 5 apresenta as considera��es finais sobre o trabalho, bem como propostas de trabalhos futuros.
% \include{capitulo2}

%% Parte p�s-textual
\backmatter

\appendix
% \include{apendice1}

% Bibliografia
% � aconselh�vel utilizar o BibTeX a partir de um arquivo, digamos "biblio.bib".
% Para ajuda na cria�?o do arquivo .bib e utiliza�?o do BibTeX, recorra ao
% BibTeXpress em www.cin.ufpe.br/~paguso/bibtexpress
%\nocite{*}
\bibliographystyle{alpha}
\bibliography{referencias}

% Inclui uma pequena nota com refer�ncia � UFPEThesis
% \colophon

%% Fim do documento
\end{document}
