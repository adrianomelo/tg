% !TEX encoding = ISO-8859-1
\documentclass[bsc]{ufpethesis}

%\university{<NOME DA UNIVERSIDADE>}
%\address{<CIDADE DA IES>}
\institute{Centro de Inform�tica}
%\department{Centro de Inform�tica}
\program{Gradua��o em Ci�ncia da Computa��o}
\majorfield{Ci�ncia da Computa��o}
\title{Normaliza��o de ontologias ALC para o uso com o LeanCop}
\date{11 de julho de 2011}
\author{Adriano Silva Tavares de Melo}
\adviser{Frederico Freitas}
%\coadviser{NOME DO(DA) CO-ORIENTADOR(A)}

\begin{document}

%% Parte pr�-textual
\frontmatter
% Folha de rosto
\frontpage
% Portada (apresenta��o)
\presentationpage
% Dedicat�ria
\begin{dedicatory}
Eu dedico esse trabalho a minha m�e, namorada, familiares e amigos.
\end{dedicatory}
% Agradecimentos
\acknowledgements
% !TEX encoding = ISO-8859-1
Gostaria de agradecer � minha m�e, que sempre acreditou em mim, que me deu condi��es de estudar, que abdicou de v�rias coisas ao decorrer de sua vida para dar prioridade aos filhos, que nos incentivou a crescer, que nos ensinou valores, que, em fim, nos amou incondicionalmente.

Agrade�o a todos da minha fam�lia que me apoiaram a estudar e a trilhar caminhos que me deixaram longe, em especial ao meu irm�o, madrinha e av�, que sempre me falam o quanto s�o orgulhosos da minha decis�o de sair da minha cidade para evoluir profissionalmente. Obrigado!

Agrade�o � minha namorada, que tornou essa jornada longe dos amigos e fam�lia mais agrad�vel e feliz.

Agrade�o a Fred, que sempre que precisei se mostrou disposto a ajudar e que me ensinou muito nos �ltimos anos.

Agrade�o a todos que me ajudaram profissionalmente e academicamente, eu n�o teria conseguido chegar at� aqui sem essa ajuda. Obrigado!

%Gostaria de agradecer primeiro a minha m�e que me deu condi��es de sair de minha cidade natal para fazer uma gradua��o, m�e, amo-te!

\begin{epigraph}[<NOTA>]{<AUTOR>}
<DIGITE AQUI A CITA��O>
\end{epigraph}

\resumo
A World Wide Web revolucionou a comunica��o entre as pessoas em todo o mundo. Com ela, o custo de enviar uma informa��o de qualquer lugar do mundo para qualquer lugar do mundo foi reduzida a praticamente a zero. Essa facilidade de acesso a informa��o global fez com que ela crescesse de forma exponencial. 
\begin{keywords}
m�todo de conex�es, l�gica de descri��o, normaliza��o
\end{keywords}

\abstract
\begin{keywords}
<DIGITE AS PALAVRAS-CHAVE AQUI>
\end{keywords}

\tableofcontents
\listoffigures
\listoftables

%% Parte textual
\mainmatter

\chapter{Introdu��o}
\label{ch:introducao}

\section{Contextualiza��o}
A Web � uma das inven��es mais revolucion�rias que o homem j� concebeu. As estimativas de penetra��o mundial, quantidade de informa��es geradas e transa��es financeiras efetuadas sobre essa plataforma s�o n�meros que v�o muito al�m das espectativas de seu criador, Tim Berners-Lee.

Berners-Lee prop�s como seria a Web em mar�o de 1989 ~\cite{BernersLee:1989}, tendo como principais atores documentos interligados atrav�s de hyperlinks. 

 
I have a dream for the Web [in which computers] become capable of analyzing all the data on the Web ? the content, links, and transactions between people and computers. A ?Semantic Web?, which should make this possible, has yet to emerge, but when it does, the day-to-day mechanisms of trade, bureaucracy and our daily lives will be handled by machines talking to machines. The ?intelligent agents? people have touted for ages will finally materialize.

\section{Organiza��o da Disserta��o}

Esta disserta��o est� dividida em seis cap�tulos. No Cap�tulo 1, � apresentada uma 
vis�o geral sobre redes neurais modulares, seus principais benef�cios e motiva��es.
Tamb�m s�o apresentados os principais objetivos desse trabalho. No Cap�tulo 2, s�o
apresentadas as etapas da constru��o de uma rede modular, bem como os principais 
m�todos da literatura utilizados em cada etapa. No Cap�tulo 3 s�o apresentadas as
propostas para decomposi��o de tarefas. No Cap�tulo 4 s�o apresentadas duas propostas 
de arquiteturas modulares, sendo uma delas obtida a partir de uma das propostas para
decomposi��o de tarefas. No Cap�tulo 5 s�o descritos os experimentos, mostrando as
bases de dados utilizadas, os resultados obtidos e as an�lises. Por fim, o Cap�tulo
6 apresenta as considera��es finais sobre o trabalho, bem como propostas de trabalhos
futuros.
% \include{capitulo2}

%% Parte p�s-textual
\backmatter

\appendix
% \include{apendice1}

% Bibliografia
% � aconselh�vel utilizar o BibTeX a partir de um arquivo, digamos "biblio.bib".
% Para ajuda na cria�?o do arquivo .bib e utiliza�?o do BibTeX, recorra ao
% BibTeXpress em www.cin.ufpe.br/~paguso/bibtexpress
%\nocite{*}
\bibliographystyle{alpha}
\bibliography{referencias}

% Inclui uma pequena nota com refer�ncia � UFPEThesis
% \colophon

%% Fim do documento
\end{document}
